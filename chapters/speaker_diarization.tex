
\chapter{Speaker Diarization in Co-channel Speech}
\label{chap:spkr_diar}

Throughout the course of this study, part of the agenda has been to acknowledge non-overlapping speech as an important component of co-channel. 
We saw that distinguishing overlap from co-channel speech expands the type of signals that are considered co-channel. 
Therefore, a close investigation of conversational speech is part of co-channel analysis. 
Among the various aspects of analyzing conversations, the one most related to speaker recognition is {\it speaker diarization}.
Speaker diarization refers to the task of automatically identifying who spoke when within an audio signal comprising of two or more speakers. 
This chapter addresses the tasks of 1) segmenting co-channel data into single-speaker excerpts and 2) clustering segments to identify which belong to the same speaker. 
The tasks are described within the context of CRSSDiar, a speaker diarization tool-kit designed to perform speaker diarization while simultaneously supporting speaker recognition and speech recognition using Kaldi\cite{kaldi}. 

CRSSDiar is a speaker diarization tool-kit developed as part of a collaboration with Chengzhu Yu, a fellow PhD student at CRSS. 
The main motivation behind developing this tool-kit was to establish an integrated end-to-end conversation analysis system that provides the capability of diarizing the signal while supporting speech recognition. 
Currently, one of the most popular speech recognition platforms used in research is Kaldi, developed by Johns Hopkins University~\cite{kaldi}. 
Existing diarization systems are implemented in different platforms and none support Kaldi I/O functions. 
Switching between platforms and API frustrates users interest in analyzing both speaker diarization and speech recognition simultaneously. 
We hope that developing a speaker diarization module on top of Kaldi will help others in CRSS with their multi-purpose research. 
Although CRSSDiar is currently in a working state, as a research platform it is still considered under development and is available for those who are interested in working on speaker diarization. 
In Sect.~\ref{sec:crssdiar_layout}, a layout of the system and its relationship with Kaldi is presented. 
A list of our modules and their Input/Output is provided. I also explain how these module interact with Kaldi data types. 
In Sect.~\ref{sec:segmentation}, the first task of speaker diarization, segmentation, is described. 
The purpose of segmentation is to split a co-channel signal into smaller chunks that contain only one speaker. 
Some segments may contain overlapped speech or no speech at all. 
Therefore, as part of segmentation, speech activity detection (SAD) and overlap detection modules are also described. 
In Sect.~\ref{sec:clustering}, the clustering (grouping) module is described. 
A state-of-the-art technique, called integer linear programming (ILP), is used in CRSSDiar to cluster segments obtained from the segmentation component~\cite{diarization_ilp}. 
Clustered groups represent segments that belong to the same speaker. 

\section{CRSSDiar Layout}
\label{sec:crssdiar_layout}


\section{Segmentation}
\label{sec:segmentation}

\section{Clustering}
\label{sec:clustering}