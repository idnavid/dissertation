
\chapter{INTRODUCTION}

Co-channel speech refers to single-channel audio signals that contain more than one speaker. 
A wide range of terms have been used to describe various aspects of co-channel speech, 
which we will clarify throughout this chapter. 
We consider both conversational speech and artificially mixed streams as co-channel. 
All signals treated in this study are single-channel recordings. 
Of all such data, a subset may have more than one ``active'' speaker at the same time, i.e. multi-speaker segments, 
which we label as ``overlapped speech''. 
Overlapped regions are segments of a co-channel signal where both speakers are simultaneously active. This categorization is summarized in Fig.~\ref{fig:cochannel_vs_overlap}.

The specifics of recording conditions are overlooked in this study. 
For example, information that relates to each speaker’s distance from the microphone and room acoustics. 
This is intentional, since most of the difficulty in dealing with co-channel speech arises 
from it being single-channel, which implies limited access to spatial 
information as well as other sources of meta-data.

There are different ways of categorizing co-channel data in terms of how it is generated. This study considers a semantic classification that focuses on the speakers' interactions with one-another. We divide co-channel data into two subgroups: 
\begin{itemize}
	\item conversational co-channel speech
	\item co-channel speech with independent parties	
\end{itemize} 

Conversational co-channel speech refers to recordings in which speakers acknowledge other parties in the recording and engage in dialog. 
Alterations of speech production by each speaker are an important artifact of conversational co-channel speech, which are the result of conscious and unconscious reactions of the foreground speaker and interferer(s) during speech especially at or around overlaps. 
Examples of such alterations are raised pitch and energy level [2]. 
The ESPN show First Take is filled with arguments between the show’s two regular sports commentators Stephen A. Smith and Skip Bayless. 
First Take is a perfect example of an exaggerated version of the above-mentioned changes in speech production. 
These changes are problematic in automatic speech applications and are considered a type of distortion. 
Consequently, the treatment will be directed towards applications that suffer the most from such alterations, predominantly speech recognition. 

Co-channel data with independent parties, are examples of co-channel data where the speakers do not interact with each other. 
Cross-talk between separate channels is considered a source of such co-channel speech. 
The main characteristic of such data is that speakers are not aware of each other and therefore do not pertain to normal conversational mannerisms. 
That includes following turn-takings rules and limiting overlapped speech. 
Artificially generated co-channel data (mixing independent channels) is another example of co-channel speech with independent parties. 
A considerable portion of this study will focus on this type of artificially generated co-channel data to analyze performance of overlap detection and also speaker recognition. 
We rely on this data since it provides the flexibility of controlling the amount of overlapped speech. 
As we will show in this chapter, conversational co-channel speech does not necessarily contain sufficient overlapped data for some of our experiments. 

\section{Background}
\label{sec:background}



\section{Approach}

The goal of this thesis is to provide tangible solutions to problems caused by co-channel speech in automatic speech technology. 
We argue that part of these issues are caused by overlapped speech (direct speech interference), which plays a significant role in making co-channel speech a difficult problem. 
The presence of overlapped speech can be detrimental to speaker diarization and speech recognition systems. 
There is no clear and unique way of labeling or transcribing overlapped segments. 
In speaker diarization, it becomes difficult to assess system performance at overlaps. 
The same goes for speech recognition. 
Aside from determining which is the ``primary'' speaker, recognizing speech at overlaps is more difficult due to interference from other speakers. 
For this and other reasons detailed in the next chapter, the first portion of this study is devoted to overlapped speech detection. 
Our approach to overlap detection will be to focus on developing signal processing techniques to detect and separate overlap from single-speaker speech. 

Although overlap is considered an important aspect of co-channel speech, in many conversational speaker recognition data, the amount of overlap in co-channel speech could be considered negligible for some tasks. 
For example, in the case of speaker recognition, we are usually interested in a specific speaker, but the audio file may contain other speakers as well. 
The standard approach in dealing with this problem is speaker diarization. 
Speaker diarization is considered a ``signal-level'' solution to this problem. 
In addition to devoting one chapter to speaker diarization, a novel approach is presented with the intention of bypassing the use of speaker diarization in the aforementioned scenario while preserving speaker recognition performance. 
This approach is to remove unwanted speaker-dependent information from latent variable subspaces generated from audio files. 
We refer to such solutions as ``subspace level''. 
We propose using two techniques to remove interfering speakers from co-channel data. 
This is performed in two different ways: 
\begin{enumerate}
\item Remove interfering speakers in the feature subspace level: i-vector subspace factorization.
\item Remove interfering speakers in the signal level: speaker diarization.
\end{enumerate}
Each of these methods will be described in a separate chapter. 
Speaker diarization, will attempt to recognize and group speech that belongs to the same speaker in a co-channel audio stream. 
While subspace factorization maps speaker-dependent models to a subspace that will only contain parameters identifying the speaker of interest (aka primary speaker). 



 