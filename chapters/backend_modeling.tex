
\chapter{FEATURE SUBSPACE MODELING}

In this chapter, we address the problem of speaker recognition for co-channel recordings. 
The difference between what we present here and every other study in this area is that we would like to bypass solutions that require removing interfering speech from the original signal, which are primarily known as speaker diarization. 
Alternatively, we are interested in modifying model parameters extracted from co-channel data in a way that would only represent the primary speaker. 
The most standard generative parameterization of speaker dependent models is called i-vector extraction~\cite{najeem_frontendanalysis}. 
These vectors (i.e., i-vectors) are latent parameters that model the covariance of speaker-dependent Gaussian mixture models (GMM) with respect to a generic speaker-independent GMM. 
In other words, i-vectors model speaker specific traits by quantifying GMM mean variabilities that belong to a certain speaker by comparing the means to universal background model (UBM) means. 
GMM means for each speaker are summarized into a single ``super-vector'' by concatenating means from all mixtures in a speaker-dependent GMM. 
This variation across super-vectors is represented by factor loading vectors in a ``total variability'' matrix. 
Factor loading coefficients provide a projection of super-vectors into a smaller subspace, i-vector subspace. 
The use of i-vectors, factor analysis in general, has become a standard way of viewing and modeling speaker specific traits in speaker recognition community. 
This type of analysis is sometimes called subspace analysis; since it reduces the dimension of GMM means to a lower dimensional subspace that only represents certain variabilities in the acoustic space. 
The goal of this chapter is to build upon this perspective in order to improve speaker recognition in co-channel signals. 
This provides the luxury of short-circuiting speaker diarization, which as we will see in the next chapter is a computationally intensive solution. 

We first describe the problem statement to develop a better understanding of how and why our approach is useful. 

\section{Problem Statement}

A main contributor to advancements in speaker recognition is the unique way in which the problem has been formulated. 
The standard speaker recognition problem used in the community is in the form of a detection problem, called speaker verification. 
The speaker verification framework, considered the ``cleanest problem'' in speech research~\cite{anintroductiontoapplicationindependentevaluationofspeakerrecognitionsystems}, was developed and improved upon over the years by the speaker recognition research community. 
Consider the problems of identifying speakers by their voice. 
In this form, the problem is extremely difficult and intractable since it implies the need for a database of all speakers and a parameterization of their voice. 
In smaller scales, identification appears to be somewhat feasible; for example identifying members of a family. 
But imagine if we are interested in larger groups, then the amount of data and number of parameters required to model speakers grows rapidly. 
Also, comparing systems becomes difficult since each system is designed to perform well on a certain group of speakers and under specific conditions. 
Keep in mind that some speakers are inherently more difficult to identify~\cite{farmmodel}.
The aforementioned issues have been addressed by reformulating speaker recognition as a verification problem. 
Where instead of identifying a speaker, we are interested in determining whether two audio samples belong to the same speaker. 
Or more accurately, whether a given audio sample belongs to a known speaker(or not). 
With this formulation, speaker verification, ``trial''\'s are evaluated independently. 
This independence has enabled the scalability of speaker recognition problems. 
One can easily imagine the applications of such a formulation. 
In user authorization systems (e.g. banking) or in forensic cases. 
Which brings us to our next discussion, forensic applications. 
Despite significant improvements in automatic speaker recognition systems over the past two decades, the use of speaker recognition systems has yet to flourish as a means for biometric verification. 
The reasons of why the criminal justice system is still largely hesitant to more widely adapt automatic speaker verification is beyond the scope of this study. 
However, I will cite an analogy presented in~\cite{reva_nist} between DNA biometric verification and voice-based verification that shows the importance of studying speaker verification in co-channel speech. 
The comparison between math problems and DNA verification is provided 
This chapter focuses more on forensic implications. 
Consider the case where we have considerable amounts of data available for a certain speaker. 
However, the recordings are co-channel and contain data from speakers other than the person of interest. 
