\documentclass[doublespacing]{utdthesis}
% For one-and-a-half spacing, use: \documentclass[halfspacing]{utdthesis}

%%% Load any desired packages in the space below.
%%% Warning: Do not load packages that change the margins, headers, or footers!
%%%
% Optional: If you want to use Times as your font, load it here.  Note that
% although package "times" should work, it may not be the best choice.  Newer
% LaTeX distributions offer "mathptmx" and "newtxtext,newtxmath" as superior
% replacements.  You should find out which is best for your LaTeX.  (If this
% sounds confusing, you probably shouldn't use Times at all.)
%\usepackage{times}
%
% Optional: If your LaTeX has microtype, use it to improve text quality:
\usepackage{microtype}
%
% Recommended: If your thesis contains math, use the AMS packages:
\usepackage{amsmath,amssymb,amsthm}
%
% Recommended: If your thesis needs to import graphics, use graphicx:
\usepackage{graphicx}
%
% Recommended: If your bibliography contains web page URLs, the url package
% improves their appearance (e.g., better line breaking):
\usepackage{url}
%
% Required: To satisfy UTD's formatting requirements for citations, use the
% "natbib" package.  (Use other citation packages at your own risk; not all
% are flexible enough to meet UTD's requirements.)  If you wish to use numeric
% citations, change "authoryear" to "numbers" below.  Use the "chicago" BibTeX
% style, which most closely matches the Turabian formatting required by UTD.
% UTD mandates a blank line between each pair of bibliography entries, so set
% \bibsep as shown below.  Finally, if you are accustomed to using \cite as
% your citation macro, point it to natbib's \citep macro as shown.
\usepackage[authoryear]{natbib}
\bibliographystyle{chicago}
\setlength{\bibsep}{12pt plus 1pt minus 1pt}
\let\cite=\citep
%
% Required: If you have any wide tables or figures that need to be typeset
% in landscape, use the rotating package:
\usepackage{rotating}
%
% Optional: If you use hyperref to auto-generate hyperlinks, always load it
% LAST since it modifies everything else.  In addition, only load hyperref if
% you use pdftex or pdflatex to generate PDFs directly.  Do NOT use it if you
% use plain tex or latex to generate a DVI file.  (If you are generating DVI
% files which you then convert to PDF, you should seriously consider switching
% to pdflatex.  The DVI format loses information because it cannot support
% modern PDF document features.  Using pdflatex to generate PDFs directly
% therefore results in PDFs of significantly higher quality.)
\usepackage{ifpdf}
\ifpdf
  \usepackage{hyperref}
\fi
%
%%% End of packages.

%%% Define all your personal macros here (if you have any).
%
\providecommand{\hyperref}[2][]{#2}

\newenvironment{exampleclasscode}
 {\parindent=1cm\begin{verse}}
 {\end{verse}}
%
%%% End of personal macro definitions.


%%% The following definitions MUST come before the document begins.
%
\author{Navid Shokouhi}
\title{Advancements in Automatic \\Speaker and Speech Processing \\in Co-channel Speech}
\thesistype{Dissertation}  % or "Thesis"
\degreefull{Doctor of Philosophy}
\degreeabbr{PhD}
\subject{Electrical Engineering}
\graduationmonth{November}
\graduationyear{2016}
\prevdegrees{BS} % comma-separated list of PREVIOUS degrees

% List committee members in order.  Mark chairpersons with a "*":
\committeemember*{John H. L. Hansen}
\committeemember{Carlos Busso}
\committeemember{Issa Panahi}
\committeemember{P. K. Rajasakeran}
%
%%% End of definitions.


%%% Beginning of actual thesis document.

\begin{document}

\frontmatter

\signaturepage
%\copyrightpage{2012} % optional

\begin{dedication} % optional
This thesis class file \\
is dedicated to ..., \\
who ...
\end{dedication}

\maketitle

\begin{acks}{November 2016} % date when thesis first submitted to committee


\end{acks}

\begin{preface} % may or may not be required, depending on thesis content
% author may insert additional preface text here
\prefacetext
% author may insert additional preface text here
\end{preface}

\begin{abstract}
This mock dissertation concerns the development and usage of a \LaTeX{}
class file that eases the task of creating UTD theses and dissertations.
The class file automatically creates margins, page headers and footers, page
numbers, paragraph parameters, title pages, and table/figure captions
consistent with the guidelines set forth by the UTD graduate school.
In contrast to many prior works, care has been taken to respect relevant
\LaTeX{} coding conventions and standards.
This helps to maximize compatibility with other \LaTeX{} packages, and
eases the incorporation of existing publication texts into a dissertation
master document.
\end{abstract}

\tableofcontents
\listoffigures % required if you have any figures
\listoftables % required if you have any tables

\mainmatter

\chapter{Introduction}
\label{c:intro}

The rest of the sample dissertation proceeds as follows:
Chapter~\ref{c:usage} details the proper usage of macros provided by the
class file.
Chapter~\ref{c:conclude} concludes the sample dissertation.
The \hyperref[a:other]{appendix} demonstrates the formatting of an appendix
(optional).

\chapter{Usage Instructions}
\label{c:usage}


\chapter{LITERATURE}
\section{temp}

Some text. 

\chapter{Conclusion}
\label{c:conclude}



\chapter*{Sample Solo Appendix}
\label{a:other}

\begin{thesisbib}
\end{thesisbib}  % <-- THIS LINE IS REQUIRED!

\begin{vita}
Navid Shokouhi
\end{vita}

\end{document}

